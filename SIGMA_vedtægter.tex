\documentclass[danish,a4paper,twocolumn]{article}
\usepackage[margin=2cm]{geometry}
\usepackage[utf8]{inputenc}
\usepackage[T1]{fontenc}
\usepackage{babel}
\usepackage{color}
\usepackage{titlesec}
%\usepackage[sc]{mathpazo}

\newcommand{\foreningen}{Sigma}

% Smarte referencer
\usepackage{smartref}
\addtoreflist{section}
\addtoreflist{subsection}
\newcommand{\longref}[1]{\sectionref{#1}, \subsectionref{#1}}

% Typografi
\setlength\parindent{0pt}
\renewcommand{\thesection}{\S\arabic{section}}
\renewcommand{\thesubsection}{Stk.~\arabic{subsection}}
\titlespacing*{\section}{0pt}{10pt}{5pt}
\titleformat{\section}{\normalfont\large\bfseries}{\thesection}{1em}{}
\titlespacing*{\subsection}{0pt}{9pt}{0pt}
\titleformat{\subsection}{\normalfont\bfseries}{\thesubsection}{1em}{}



% Ordeling
\hyphenation{uni-ver-si-tet fler-tal}

\begin{document}
\title{\vspace{-2ex}Statut for \foreningen\vspace{-5ex}}
\date{27. Marts 2021}
\maketitle

\section{Formål og tilhørsforhold}
\subsection{}Foreningens navn er \foreningen.
\subsection{}Hjemstedet er på hovedområdet Natural Science
ved Aarhus Universitet i Aarhus Kommune.

\subsection{}Sigma har til formål at repræsentere studerende ved
uddannelserne under følgende instituttter og centre:
\begin{itemize}
        \item Institut for matematik
        \item Institut for Fysik og Astronomi
        \item Institut for datalogi
        \item Institut for kemi
        \item iNANO
\end{itemize}
Denne gruppe af uddannelser betegnes herefter Faggruppen.
\subsection{}Sigma har til formål at virke til gavn for de studerende i Faggruppen ved at varetage deres fælles faglige, økonomiske og sociale interesser. Arbejdet er uafhængigt af politiske og økonomiske interesser.
\subsection{}Sigma udgør Faggruppens repræsentation ved Studenterrådet ved Aarhus Universitet og lader sig derfor repræsentere i Fællesrådet.

\section{Struktur og tegning}
\subsection{}Sigmas øverste myndighed er generelforsamlingen.
\subsection{}Sigmas daglige ledelse er Sigma-møderne.
\subsection{}Såfremt en indvalgt ikke længere er studerende under Faggruppen er denne ikke længere indvalgt.
\subsection{}Tilstedeværende til et Sigma møde kan til enhver tid indvælges af et flertal af tilstedeværende Sigma indvalgte.
\subsection{}Sigma tegnes af formanden og kassereren.

\section{Økonomi og revision}
\subsection{}Kassereren fører løbende regnskab og tilsyn med Sigmas økonomi.
\subsection{}Kassereren har beføjelse til at administrere
foreningens økonomi på egen hånd. Udlæg der overstiger et beløb på 5000
kr. skal dog vedtages på et Sigma-møde jvf. \ref{par:sigmamdr}.

\subsection{}På generalforsamlingen fremlægger kassereren det reviderede regnskab til godkendelse. Regnskabet medsendes dagsordenen som bilag.
\subsection{}Revisorerne fremlægger deres bemærkninger til regnskabet i forbindelse med kassererens fremlæggelse.
\subsection{}Kassereren er forpligtiget til at bistå revisorerne med
fremskaffelse af kontoudtog og øvrige bilag til regnskabet.

\section{Studentermøder}\label{par:studmdr}
\subsection{}På Studentermøderne har alle studerende ved Faggruppen
tale- og stemmeret og herudover har repræsentanter fra Fællesrådet og
andre studenterpolitiske organisationer ved Natural Science taleret.

\subsection{}Studentermødet kan herudover vælge at lade andre deltagere end de ovennævnte have taleret.
\subsection{}Studentermødet ledes af en dirigent, der vælges af og blandt de tilstedeværende.
\subsection{}Det er dirigentens pligt at konstatere hvorvidt der er rettidigt indkaldt til mødet og dermed om dette er beslutningsdygtigt.
\subsection{}Der vælges en referent blandt de, ved mødets start, indvalgte i Sigma.
\subsection{}Et Studentermøde kan indkaldes af:
\begin{itemize}
        \item en eller flere af Sigmas indvalgte.
        \item en eller flere studenterrepræsentanter fra Fagruppen, der er
            indvalgt i udvalg, råd og/eller nævn ved Natural Science.
        \item 10 eller flere studerende i Fagruppen.
\end{itemize}
\subsection{}\label{stk:studmd-rettidig}Der er indkaldt rettidigt til et Studentermøde såfremt indkaldelse samt dagsorden er offentliggjort senest en uge før afholdelsen.
\subsection{}En dagsorden for Studentermøder skal indeholde følgende punkter:
\begin{enumerate}
        \item Formalia
        \begin{enumerate}
                \item Valg af dirigent og referant
                \item Dirigenten konstaterer om mødet er beslutningsdygtigt
        \end{enumerate}
        \item Eventuelt
\end{enumerate}
\subsection{}Beslutninger træffes ved almindeligt flertal, dog kan der ikke træffes beslutninger under punktet Eventuelt.

\section{Sigma-møder}\label{par:sigmamdr}
\subsection{}Indvalgte i Sigma har stemme- og taleret.
\subsection{}Samtlige studerende under Faggruppen har stemme- og taleret. Stemmeretten bortfalder dog såfremt halvdelen eller flere af de fremmødte Sigma indvalgte ønsker det. Sådanne ønsker skal udtrykkes før stemmeafgivelse.
\subsection{}Sigma-møder kan vælge at lade andre end de ovennævnte have taleret ved mødet.
\subsection{}Beslutninger foretages ved almindeligt flertal.
\subsection{}Sigma-møder ledes af en dirigent, der vælges af og blandt de tilstedeværende.
\subsection{}Det er dirigentens pligt at konstatere hvorvidt der er rettidigt indkaldt til mødet og dermed om dette er beslutningsdygtigt.
\subsection{}Der er indkaldt rettidigt til et Sigma-møde såfremt indkaldelse samt dagsorden er offentliggjort senest to hverdage før afholdelse.
\subsection{}Dagsordenen skal indeholde følgende punkter:
\begin{enumerate}
        \item Valg af dirigent og referant
        \item Meddelelser fra
        \begin{itemize}
                \item Udvalg, råd og nævn hvor Faggruppen er repræsenteret
                \item Fællesrådet
                \item Eventuelle udvalg under Sigma
        \end{itemize}
        \item Næste møde
        \item Eventuelt
\end{enumerate}
\subsection{}Der kan optages ekstra punkter på dagsordenen/ændres i rækkefølgen under mødet såfremt ingen Sigmas indvalgte modsætter sig dette.
\subsection{}Referatet af Sigma-mødet offentliggøres inden afholdelse af næste møde, dog senest efter en uge.
 
\section{Generalforsamlingen}
\subsection{}Generalforsamlingen afholdes i starten af 4. kvartal ved indkaldelse fra formanden, der kan dog indkaldes til ekstraordinær generalforsamling hvis $\frac{2}{3}$ af de almene indvalgte ønsker det.
\subsection{}Samtlige studerende under Faggruppen har stemme- og taleret. Dog er stemmeretten til statutændringer forbeholdt formand, kasserer og almene indvalgte.
\subsection{}Generalforsamlingens dagsorden skal indeholde følgende punkter:
\begin{enumerate}
\item Formalia
  \begin{itemize}
  \item Valg af dirigent og referant
  \item Dirigenten konstaterer om mødet er beslutningsdygtigt
  \item Valg af stemmetællere
  \end{itemize}
\item Beretninger fra afgående
  \begin{itemize}
  \item Sigma
  \item Kassereren
  \item Studenterrådet
  \end{itemize}
\item Afgang af tidligere indvalgte
\item Valg til ansvarsposter
  \begin{itemize}
      \item Valg af formand
      \item Valg af kasserer
      \item Valg af revisorer
  \end{itemize}
\item Valg af almene indvalgte
\item Statutændringer
\item Dato for næste Sigma møde
\item Eventuelt
\end{enumerate}
\subsection{}Der er indkaldt rettidigt såfremt indkaldelse samt dagsorden er offentliggjort senest to uger før afholdelse. Mødet er kun beslutningsdygtigt såfrem der er indkaldt rettidigt.
\subsection{} Ved afgang af tidligere indvalgte afgår formand, kasserer og alle almene indvalgte.
\subsection{}Sigmas beretning skal indeholde oversigt over afholdte arrangementer i Sigma regi samt relevante politiske aktiviteter.
\subsection{}Kassererens beretning skal indeholde det reviderede og af revisorerne godkendte regnskab. Revisorernes bemærkninger skal ligeledes fremgå.
\subsection{}Studenterrådets beretning skal indeholde en oversigt over de vigtigste aktiviteter i Studenterrådet, der har relevans for Faggruppen.
\subsection{}Kandidater opstilles på mødet blandt de tilstedeværende
studerende fra Faggruppen eller ved skriftlig tilstedeværelse.
\subsection{}Valget afholdes som plenumvalg ved håndsoprækning, hvis det ønskes kan der afholdes hemmelig afstemning.
\subsection{}Dirigenten orienterer Fællesrådet om nye indvalgte til formands- og kassererposten skriftligt.
\subsection{}Formanden vælges blandt de tilstedeværende med stemmeret ved almindeligt flertal. 
\subsection{}Kassereren vælges blandt de tilstedeværende med stemmeret ved almindeligt flertal.
\subsection{}Der vælges to revisorer, hvoraf kun den ene kan være indvalgt i Sigma.
\subsection{}Ingen person kan bestride mere end en af ovenstående poster beskrevet i stk. 12-14

\section{Repræsentation i udvalg, råd og nævn}
\subsection{}Sigma skal foranledige opstilling af kandidater til udvalg, råd og
nævn under Natural Science, hvor Faggruppen er repræsenteret (herefter Organerne).

\subsection{}Kanditater valgt til Organerne forpligtiger sig til at
\begin{itemize}
        \item holde Sigma løbende orienteret om deres virke i Organet
        \item følge linjerne i Sigmas arbejde
        \item følge øvrige retningslinjer pålagt af Sigma- eller Studentermøder
\end{itemize}

\section{Statutændringer}\label{par:aendringer}
\subsection{}Statutændringer kan kun vedtages på en generalforsamling hvor ændringforslaget opnår mindst $\frac{2}{3}$ tilslutning. Stavefejl og andre mindre fejl kan dog til enhver tid rettes, dog må meningen med statutten ikke ændres og det skal informeres om og godkendes af et flertal på efterfølgende Sigma møde.

\subsection{}Statutændringer skal være indsendt mindst 2 uger før afholdelse af generalforsamlingen.


\subsection{}Denne statut er vedtaget på Studentermødet d. 27. Marts 2021 og erstatter statutten af 16. februar 2020

\vspace{2cm}
$\overline{\emph{Jes Enok Steinmüller, Formand}}$

\vspace{1cm}
$\overline{\emph{Rune , Kasserer}}$
\end{document}

%%% Local Variables:
%%% mode: latex
%%% TeX-master: t
%%% End:
