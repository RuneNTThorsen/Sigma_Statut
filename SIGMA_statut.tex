\documentclass[danish,a4paper,twocolumn]{article}
\usepackage[margin=2cm]{geometry}
\usepackage[utf8]{inputenc}
\usepackage[T1]{fontenc}
\usepackage[danish]{babel}
\usepackage{color}
\usepackage{titlesec}
%\usepackage[sc]{mathpazo}

\newcommand{\foreningen}{Sigma}
\newcommand{\fakultetet}{Faculty of Natural Sciences}

% Smarte referencer
\usepackage{smartref}
\addtoreflist{section}
\addtoreflist{subsection}
\newcommand{\longref}[1]{\sectionref{#1}, \subsectionref{#1}}

% Typografi
\setlength\parindent{0pt}
\renewcommand{\thesection}{\S\arabic{section}}
\renewcommand{\thesubsection}{Stk.~\arabic{subsection}}
\titlespacing*{\section}{0pt}{10pt}{5pt}
\titleformat{\section}{\normalfont\large\bfseries}{\thesection}{1em}{}
\titlespacing*{\subsection}{0pt}{9pt}{0pt}
\titleformat{\subsection}{\normalfont\bfseries}{\thesubsection}{1em}{}



% Ordeling
\hyphenation{uni-ver-si-tet fler-tal}
\renewcommand{\danishhyphenmins}{22}

\begin{document}
\title{\vspace{-2ex}Statut for \foreningen\vspace{-5ex}}
\date{7. december 2022}
\maketitle

\section{Formål og tilhørsforhold}
\subsection{}Foreningens navn er \foreningen.
\subsection{}Hjemstedet er \fakultetet ved Aarhus Universitet.

\subsection{}\foreningen har til formål at repræsentere studerende ved uddannelserne tilbudt af følgende instituttter og centre:
\begin{itemize}
        \item Institut for Matematik
        \item Institut for Fysik og Astronomi
        \item Institut for Datalogi
        \item Institut for Kemi
        \item Interdisciplinary Nanoscience Center
\end{itemize}
Denne gruppe af uddannelser betegnes herefter Faggruppen.
\subsection{}\foreningen har til formål at virke til gavn for de studerende i Faggruppen ved at varetage deres fælles faglige, økonomiske og sociale interesser. Arbejdet er uafhængigt af politiske og økonomiske interesser.
\subsection{}\foreningen udgør Faggruppens repræsentation ved Studenterrådet ved Aarhus Universitet og lader sig derfor repræsentere i Fællesrådet.

\section{Struktur og tegning}
\subsection{}\foreningen s øverste myndighed er generelforsamlingen.
\subsection{}\foreningen s daglige ledelse er møderne i \foreningen.
\subsection{}Såfremt en indvalgt ikke længere er studerende under Faggruppen er denne ikke længere indvalgt.
\subsection{}Tilstedeværende til et møde i \foreningen kan til enhver tid indvælges af et flertal af tilstedeværende indvalgte i \foreningen.
\subsection{}\foreningen tegnes af formanden og kassereren.

\section{Underforening}
\subsection{}Foreningen "Institut for Fysik \& Astronomi studenterrepræsentation", herefter tiltalt ISR, er en underforening af \foreningen.
\subsection{}Medlemmer af ISR er ikke nødvendigvist medlem af \foreningen og \foreningen kan derfor ikke stille de samme krav, som over for egne medlemmer.
\subsection{}Som underforening af \foreningen forpligtiger den ansvarlige i ISR sig på at sende referater til \foreningen og regelmæssigt underrette formanden for \foreningen om status i ISR.
\subsection{}Skulle ISR blive inaktiv og/eller opløses, forpligtiger \foreningen sig på at føre opsyn med muligheden for at genetablere et fungerende ISR. \foreningen kan udpege en fungerende ansvarlig for ISR indtil en sådan kan vælges af ISR selv. \foreningen forpligtiger sig til at holde ledelsen på institut for Fysik \& Astronomi underrettet om en sådan givet- eller lignende situation.
\subsection{}Udtrædelse af ISR fra \foreningen skal godkendes på et møde i ISR, et studentermøde på Institut for  Fysik \& Astronomi og et møde i \foreningen med 2/3 flertal i alle tilfælde.
\subsection{}Til et studentermøde om udtrædelse af ISR, må kun dette emne behandles. Kun fysik- og astronomistuderende vil have stemmeret.

\section{Økonomi og revision}
\subsection{}Kassereren fører løbende regnskab og tilsyn med \foreningen s økonomi.
\subsection{}Kassereren har beføjelse til at administrere foreningens økonomi på egen hånd. Udlæg der overstiger et beløb på 5000 kr. skal dog godkendes på et møde i \foreningen jvf. \ref{par:sigmamdr}.
\subsection{}Til generalforsamlingen fremlægger kassereren det reviderede regnskab til godkendelse, jvf. \ref{par:GF}. Regnskabet medsendes dagsordenen som bilag.
\subsection{}Revisorerne fremlægger deres bemærkninger til regnskabet i forbindelse med kassererens fremlæggelse til generalforsamlingen, jvf. \ref{par:GF}.
\subsection{}Kassereren er forpligtiget til at bistå revisorerne med fremskaffelse af kontoudtog og øvrige bilag til regnskabet.

\section{Studentermøder}\label{par:studmdr}
\subsection{}På Studentermøderne har alle studerende ved Faggruppen tale- og stemmeret.

\subsection{}Studentermødet kan herudover vælge at lade andre deltagere end de ovennævnte have taleret.
\subsection{}Studentermødet ledes af en dirigent, der vælges af og blandt de tilstedeværende.
\subsection{}Det er dirigentens pligt at konstatere hvorvidt der er rettidigt indkaldt til mødet og dermed om dette er beslutningsdygtigt.
\subsection{} Ved mødets start vælges der en referent blandt de indvalgte i \foreningen.
\subsection{}Et Studentermøde kan indkaldes af:
\begin{itemize}
        \item en eller flere af \foreningen s indvalgte.
        \item en eller flere studenterrepræsentanter fra Fagruppen, der er
            indvalgt i udvalg, råd og/eller nævn ved \fakultetet.
        \item 10 eller flere studerende i Faggruppen.
\end{itemize}
\subsection{}\label{stk:studmd-rettidig}Der er indkaldt rettidigt til et Studentermøde såfremt indkaldelse samt dagsorden er offentliggjort senest en uge før afholdelsen.
\subsection{}En dagsorden for et Studentermøde skal indeholde følgende punkter:
\begin{enumerate}
        \item Formalia
        \begin{enumerate}
                \item Valg af dirigent og referent
                \item Dirigenten konstaterer om mødet er beslutningsdygtigt
        \end{enumerate}
        \item Eventuelt
\end{enumerate}
\subsection{}Beslutninger træffes ved almindeligt flertal, dog kan der ikke træffes beslutninger under punktet Eventuelt.

\section{Møder i \foreningen}\label{par:sigmamdr}
\subsection{}Indvalgte i \foreningen har stemme- og taleret.
\subsection{}Samtlige studerende under Faggruppen har stemme- og taleret. Stemmeretten bortfalder dog såfremt halvdelen eller flere af de fremmødte i \foreningen indvalgte ønsker det. Sådanne ønsker skal udtrykkes før stemmeafgivelse.
\subsection{}Mødedeltagerne til et møde i \foreningen kan vælge at lade andre end de ovennævnte have taleret ved mødet.
\subsection{}Beslutninger foretages ved almindeligt flertal.
\subsection{}Møder i \foreningen ledes af en dirigent, der vælges af- og blandt de tilstedeværende.
\subsection{}Det er dirigentens pligt at konstatere hvorvidt der er rettidigt indkaldt til mødet og dermed om dette er beslutningsdygtigt.
\subsection{}Der er indkaldt rettidigt til et møde i \foreningen såfremt indkaldelse samt dagsorden er offentliggjort senest to hverdage før afholdelse.
\subsection{}Dagsordenen skal indeholde følgende punkter:
\begin{enumerate}
        \item Valg af dirigent og referent
        \item Meddelelser fra
        \begin{itemize}
                \item Udvalg, råd og nævn hvor Faggruppen er repræsenteret
                \item Fællesrådet
                \item Eventuelle udvalg under \foreningen
        \end{itemize}
        \item Næste møde
        \item Eventuelt
\end{enumerate}
\subsection{}Der kan optages ekstra punkter på dagsordenen/ændres i rækkefølgen under mødet såfremt ingen af \foreningen s indvalgte modsætter sig dette.
\subsection{}Referatet af mødet i \foreningen offentliggøres inden afholdelse af næste møde, dog senest efter en uge.
 
\section{Generalforsamlingen}\label{par:GF}
\subsection{}Generalforsamlingen afholdes i starten af 4. kvartal ved indkaldelse fra formanden. Der kan dog indkaldes til ekstraordinær generalforsamling hvis $\frac{2}{3}$ af de almene indvalgte ønsker det.
\subsection{}Samtlige studerende under Faggruppen har stemme- og taleret. Dog er stemmeretten til statutændringer forbeholdt formand, kasserer og almene indvalgte.
\subsection{}Generalforsamlingens dagsorden skal indeholde følgende punkter:
\begin{enumerate}
\item Formalia
  \begin{itemize}
  \item Valg af dirigent og referent
  \item Dirigenten konstaterer om mødet er beslutningsdygtigt
  \item Valg af stemmetællere
  \end{itemize}
\item Beretninger fra afgående
  \begin{itemize}
  \item \foreningen
  \item Kassereren
  \item Studenterrådet
  \end{itemize}
\item Afgang af tidligere indvalgte
\item Valg til ansvarsposter
  \begin{itemize}
      \item Valg af formand
      \item Valg af kasserer
      \item Valg af revisorer
  \end{itemize}
\item Valg af almene indvalgte
\item Statutændringer
\item Dato for næste møde i \foreningen
\item Eventuelt
\end{enumerate}
\subsection{}Der er indkaldt rettidigt såfremt indkaldelse samt dagsorden er offentliggjort senest to uger før afholdelse. Mødet er kun beslutningsdygtigt såfrem der er indkaldt rettidigt.
\subsection{} Ved afgang af tidligere indvalgte afgår formand, kasserer og alle almene indvalgte.
\subsection{}\foreningen s beretning skal indeholde oversigt over afholdte arrangementer i regi af \foreningen samt relevante politiske aktiviteter.
\subsection{}Kassererens beretning skal indeholde det reviderede- og af revisorerne godkendte regnskab. Revisorernes bemærkninger skal ligeledes fremgå.
\subsection{}Studenterrådets beretning skal indeholde en oversigt over de vigtigste aktiviteter i Studenterrådet, der har relevans for Faggruppen.
\subsection{}Kandidater opstilles på mødet blandt de tilstedeværende studerende fra Faggruppen eller ved skriftlig tilstedeværelse.
\subsection{}Valget afholdes som plenumvalg ved håndsoprækning. Hvis det ønskes kan der afholdes hemmelig afstemning.
\subsection{}Dirigenten orienterer Fællesrådet om nye indvalgte til formands- og kassererposten skriftligt.
\subsection{}\label{stk:Formanden}Formanden vælges blandt de tilstedeværende med stemmeret ved almindeligt flertal. 
\subsection{}Kassereren vælges blandt de tilstedeværende med stemmeret ved almindeligt flertal.
\subsection{}\label{stk:revisorer}Der vælges to revisorer, hvoraf kun den ene kan være indvalgt i \foreningen.
\subsection{}Ingen person kan bestride mere end en af ovenstående poster beskrevet i \ref{stk:Formanden} til og med \ref{stk:revisorer}.

\section{Repræsentation i udvalg, råd og nævn}
\subsection{}\foreningen skal foranledige opstilling af kandidater til udvalg, råd og nævn under \fakultetet, hvor Faggruppen er repræsenteret (herefter Organerne).

\subsection{}Kandidater valgt til Organerne forpligtiger sig til at
\begin{itemize}
        \item holde \foreningen løbende orienteret om deres virke i Organet
        \item følge linjerne i \foreningen s arbejde
        \item følge øvrige retningslinjer pålagt af \foreningen- eller Studentermøder
\end{itemize}

\section{Statutændringer}\label{par:aendringer}
\subsection{}Statutændringer kan kun vedtages på en generalforsamling hvor ændringforslaget opnår mindst $\frac{2}{3}$ tilslutning. Stavefejl og andre mindre fejl kan dog til enhver tid rettes, dog må meningen med statutten ikke ændres og det skal informeres om og godkendes af et flertal på efterfølgende møde i \foreningen.

\subsection{}Statutændringer skal være indsendt mindst 2 uger før afholdelse af generalforsamlingen.

\vspace{2cm}
Denne statut er vedtaget til generalforsamlingen d. 7. december 2022 og erstatter statutten af 27. marts 2021

\vspace{0.5cm}
$\overline{\emph{Jes Enok Steinmüller, Formand}}$

\vspace{0.1cm}
$\overline{\emph{Rune Naujokat Troelsgaard Thorsen, Kasserer}}$
\end{document}

